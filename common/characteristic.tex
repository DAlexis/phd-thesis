
{\actuality} 
Диссертационная работа совмещает теоретическое и экспериментальное исследование процессов инициации молниевого разряда, стримерно-лидерного перехода, особенностей распределения молний в средней полосе России и закономерностей повторения стадии обратного удара.

Одним из хорошо известных, но недостаточно изученных явлений, сопровождающих молнию, является повторение стадии обратного удара. Для значительной части отрицательных разрядов типа облако-земля характерно повторение главной стадии молнии один или несколько раз \autocite{Rakov-PhysicsOfLightning}. Повторные компоненты молний имеют исключительную важность для задач молниезащиты. Из-за мультипликативности обратного удара импульс максимального тока может повторяться многократно за короткий промежуток времени, выводя из строя устройства, не рассчитанные на многократные перегрузки. В настоящее время существуют лишь качественные теории, объясняющие механизм возникновения повторных обратных ударов \autocite{RakovUman2005}, что обуславливает актуальность натурных исследования таких событий и статистическая обработка полученных данных.

Универсальным и мощным средством натурного наблюдения молний являются грозопеленгационные системы (ГПС) "--- аппаратно-программные комплексы, определяющие координаты и точное время молниевых вспышек на определённой территории. Помимо исследования молний, такие системы необходимы для широкого круга прикладных задач: мониторинга молниевой активности, краткосрочного прогноза быстроразвивающихся конвективных явлений, изучения климатологии молний, обеспечения безопасности движения воздушных судов, предупреждения развития опасных атмосферных событий \autocite{Matveev1984}. Грозопеленгационные сети, совместно аэроэлектрическими наблюдениями \autocite{Anisimov2013}, являются основой для задачи усвоения геоэлектрических метеоданных, исследований глобальной электрической цепи и климатологии молнии \autocite{Mareev2010,Williams2014}.

В рамках данного диссертационного исследования разработана с нуля и введена в эксплуатацию региональная грозопеленгационная система OpenLDS. Проведено исследование точности системы и определена её погрешность. На основе показаний ГПС за конвективные сезоны 2014-2017\,г. исследована зависимость количества гроз от типа местности. Подтверждено наличие городского эффекта над Н.\,Новгородом. Исследованы статистические характеристики множественных обратных ударов, и выдвинуты гипотезы, объясняющие полученные закономерности.

Одним из центральных вопросов исследования атмосферного электричества является описание процессов инициации и развития молнии. Несмотря на существенный прогресс физики газового разряда во второй половине~XX века, современные модели молний носят скорее качественный характер~\autocite{Rakov-PhysicsOfLightning}, и базовые механизмы до сих пор вызывают разногласия. В данной диссертационной работе совмещается моделирование молний из первых принципов с натурными наблюдениями посредством разработанной системы грозопеленгации.

Инициация молниевого разряда является одной из самых удивительных загадок атмосферного электричества и возглавляет список десяти наиболее важных нерешенных проблем, связанных с исследованием молниевого разряда \autocite{Dwyer2014}. До сих пор не существует единой точки зрения на то, как в грозовом облаке формируется достаточно длинный для поддержания собственного развития лидерный, максимальная напряженность электрического поля в котором на порядок ниже поля пробоя воздуха \autocite{Marshall1995}.

На текущий момент наибольшее влияние имеет несколько основных теории зарождения молнии. Первая "--- инициация стримеров с гидрометеоров \autocite{Loeb1966, Phelps1974, Griffiths1976}. Недостаток данного механизма состоит в том, что для обеспечения устойчивого развития стримерной системы необходимо наличие либо области с электрическим полем, превосходящим максимально наблюдаемые облачные поля \autocite{Griffiths1976}, либо гидрометеора с аномально большим аспектным отношением \autocite{Dubinova2015}, либо "--- предварительной ионизации \autocite{Sadighi2015, Cai2017}.

Вторая гипотеза основана на предложенном Гуревичем \autocite{Gurevich1992} механизме пробоя на высокоэнергичных убегающих электронах, эффективная сила трения для которых убывает в интервале энергий от 0.1\,кэВ до 1\,МэВ. При подходящем месте возникновения затравочных убегающих электронов, появляющихся под действием ионизации нейтральных молекул частицами космических лучей, может возникнуть электронная лавина. Предполагается, что пробой на убегающих электронах способен создать плазменное пятно, поляризация на границах которого приводит к существенному усилению поля \autocite{Gurevich1999}. Несмотря на то, что максимальные измеренные поля в грозовом облаке достаточны для развития пробоя на убегающих электронах, необходимая протяженность области сильного поля достигает километра \autocite{Gurevich2001}, что противоречит данным наблюдений. Данный механизм модифицируется в работе \autocite{Dwyer2005} за счёт позитронной петли положительной обратной связи, что ослабляет требования на пространственный масштаб электрического поля.

Еще один механизм, соединяющий идею пробоя на убегающих электронах с инициацией положительных стримерных систем с поверхности гидрометеоров, предложен Петерсеном \autocite{Petersen2008}. Предполагается, что пробой на убегающих электронах создаёт предварительную ионизацию, что необходимо для возникновения положительных стримеров с гидрометеоров, поляризованных локальным полем. Возникающий пучок положительных стримеров накапливает отрицательный заряд в точке своего основания, локально усиливая поле и создавая условия для возникновения противоположно направленных отрицательных стримеров. Отрицательные стримеры теперь уже биполярной стримерной системы, прогреваясь, формируют канал пространственного лидера и, участвуя в процессе разделения заряда, усиливают поле на периферии разрядной структуры, провоцируя появление вторичной системы положительных стримеров, развивающихся подобным образом. Затем, положительные стримеры вторичной системы сливаются с отрицательными стримерами первичной, создавая единый канал, прогреваемый токами выравнивания потенциала. В результате многократного повторения данного процесса перекрывающаяся и сливающаяся цепочка биполярных стримерных систем формирует канал лидера молнии.

В работе \autocite{Iudin2017} был предложен принципиально новый механизм зарождения молнии, основанный на индуцированном шумом кинетическом переходе, происходящем в стохастическом поле заряженных гидрометеоров. Неравновесный фазовый переход порождает пятна ионной плазмы с линейными размерами, достигающими нескольких дециметров, и временем жизни порядка нескольких десятков миллисекунд. Подчеркивается, что резкий рост ионной проводимости происходит в экспоненциально редких компактных областях пространства на фоне исчезающе малых изменений средней проводимости среды. В результате поляризации, обусловленной крупномасштабным электрическим полем грозы, поле на концах плазменных пятен усиливается до величины, достаточной для инициации положительных стримеров. По мере роста концентрации пятен ионной плазмы, коллективная динамика положительных стримерных систем обеспечивает появление лидерного канала в соответствие с качественной картиной описанных ранее сценариев Леба и Петерсона.

Для подтверждения той или иной гипотезы инициации молниевого разряда необходимо построение численной модели, основанной на идеях гипотезы. Модель должна воспроизводить все основные наблюдаемые процессы как качественно, так и количественно. Следует заметить, что в силу высокой геометрической сложности, а также большого разброса пространственных и временных масштабов, построение аналитической модели процесса инициации  молнии, претендующей более чем на качественное описание, невозможно.

В данном диссертационном исследовании представлена мелкомасштабная транспортная модель формирования древа электрического разряда в грозовом облаке, реализованная в виде программного пакета. Модель обладает рядом инновационных особенностей, существенно отличающих её от целого ряда аналогичных работ. К ним относятся отсутствие привязки к пространственной сетке, беспрецедентно высокое пространственно-временное разрешение, учет асимметрии развития положительных и отрицательных стримеров, учет временной эволюции параметров разрядных каналов и параметризация стримерно-лидерного перехода, сформулированная в терминах температуры канала и опирающаяся на хорошо изученный механизм ионизационно-перегревной неустойчивости. В рамках применяемого подхода хорошо проводящий и разогретый лидерный канал формируется за счёт объединения токов десятков тысяч стримерных каналов, изначально обладающих пренебрежимо малой проводимостью и температурой канала, не отличающейся от температуры окружающей среды. Модельное билидерное древо имеет электродинамические характеристики, промежуточные между лабораторной длинной искрой и развитой молнией. Морфологические характеристики и электрические параметры разрядного древа зарождающегося модельного лидера молнии подтверждаются современными данными о развитии молниевого разряда.

%\ifsynopsis
%Этот абзац появляется только в~автореферате.
%Для формирования блоков, которые будут обрабатываться только в~автореферате,
%заведена проверка условия \verb!\!\verb!ifsynopsis!.
%Значение условия задаётся в~основном файле документа (\verb!synopsis.tex! для
%автореферата).
%\else
%Этот абзац появляется только в~диссертации.
%Через проверку условия \verb!\!\verb!ifsynopsis!, задаваемого в~основном файле
%документа (\verb!dissertation.tex! для диссертации), можно сделать новую
%команду, обеспечивающую появление цитаты в~диссертации, но~не~в~автореферате.
%\fi

% {\progress}
% Этот раздел должен быть отдельным структурным элементом по
% ГОСТ, но он, как правило, включается в описание актуальности
% темы. Нужен он отдельным структурынм элемементом или нет ---
% смотрите другие диссертации вашего совета, скорее всего не нужен.

{\aim} данной работы является комплексное, экспериментально-теоретическое исследование процессов инициации, развития и повторения молнии. Экспериментальная часть подразумевает создание необходимого инструмента (грозопеленгационной системы) и анализ статистических характеристик молниевых разрядов в регионе. Теоретическая часть включает в себя разработку модели инициации молниевого разряда от момента возникновения первых дециметровых проводящих областей до устойчивого развития билидера.

Для экспериментального исследования молниевой активности, а также изучения мультипликативности обратного удара поставлены и решены {\tasks}:
\begin{enumerate}
	\item Изучение существующих грозопеленгационных систем и методов пеленгаци молний.
	\item Разработка региональной грозопеленгационной системы. Создание необходимого программного обеспечения с открытым исходным кодом под названием OpenLDS.
	\item Верификация точности позиционирования молниевых разрядов разработанной системы, сравнение с показаниями других ГПС.
	\item Анализ данных грозопеленгационной системы, полученных за конвективные сезоны 2014--2017\,гг.
\end{enumerate}

Для моделирования процесса инициации молнии и валидации модели поставлены и решены следующие {\tasks}:
\begin{enumerate}
	\item Изучение существующих подходов к моделированию процесса инициации молнии.
	\item Разработка транспортной самоорганизующейся модели инициации молнии из первых принципов.
	\item Создание программного пакета, выполняющего расчёты.
	\item Настройка параметров модели, проведение численных экспериментов. Сравнение результатов моделирования с экспериментальными данными. Анализ полученных результатов.
\end{enumerate}


{\novelty}

Разработанная грозопеленгационная система OpenLDS обладает наибольшей чувствительностью и точностью на территории Нижегородской области среди всех систем, данные которых доступны для научного изучения. OpenLDS "--- единственная известная автору ГПС с открытым исходным кодом. 

Впервые проведены исследования мультипликативности обратного удара в средней полосе России, основанные на данных грозопеленгационной системы на основе выборки мощностью более 1,5\,млн. молний за 4\,конвективных сезона. 

Транспортная самоорганизующаяся модель молнии, представленная в работе, обладает рядом уникальных особенностей по сравнению с моделями, известными автору. К числу таких особенностей относятся отсутствие пространственной сетки, неограниченная степень ветвления дерева разряда и произвольная ориентация проводников, беспрецедентно высокое пространственно-временное разрешение, учет асимметрии развития положительных и отрицательных стримеров, учет временной эволюции параметров разрядных каналов и самосогласованное моделирование стримерно-лидерного перехода. Модель является развитием работы \autocite{IudinRakov2017}, однако реализована с нуля и базируется на новых принципах.


{\influence}
Разработанная региональная ГПС OpеnLDS имеет широкое практическое и фундаментальное применение вне контекста данной работы, являясь самостоятельным инструментом получения информации о молниях на покрываемой территории (см. раздел \ref{sec:lds-intro}). Полученные статистические характеристики повторных обратных ударов позволяют расширить понимание процессов, предваряющих и сопровождающих развитие стреловидного лидера, а также механизмов сбора заряда в облаке. 

В рамках диссертационного исследования впервые предложена численная модель процесса инициации молнии начиная от образования первых дециметровых проводящих областей и заканчивая устойчивым развитием билидера. Модель верифицирована через воспроизводимые качественные эффекты, а также через макроскопические параметры, известные из натурных наблюдений. 

Построенная модель доказывает возможность описания процесса превращения дециметровых проводящих областей в устойчивый билидер исключительно за счёт коллективных эффектов усиления поля и нагрева стримерных пучков под действием электростатической индукции. Привлечение гидрометеоров, пробоя на убегающих электронах, а также позитронной положительной обратной связи не требуется на данном этапе развития разряда. Модель согласуется с принципами, предложенными Д.\,И.~Иудиным в работах \autocite{Iudin2017,IudinRakov2017}. Модель может быть использована для расчёта излучения в процессе инициации молнии.


{\methods}
В основе работы региональной грозопеленгационной системы лежат разностно-дальномерный метод позиционирования разрядов и метод пересечения пеленгов. Программное обеспечение ГПС разработано автором и реализовано преимущественно на языке C++ с применением современных технологий программирования и библиотек. Связь между пеленгаторами и центральным сервером ГПС осуществляется через сеть Интернет. Для анализа данных ГПС применены базовые методы математической статистики и разработано вспомогательное ПО на языке Python.

В основе модели инициации молнии лежит описание электрической цепи в виде динамического графа, вложенного в трехмерное пространство. Для описания газоразрядных процессов в элементах цепи применены параметризации процесса ионизации и прилипания электронов, а также параметризация ионизационно-перегревной неустойчивости. Расчёт электрических полей выполняется в приближении электростатики. Для выполнения расчётов разработан оригинальный программный пакет на языке C++ с применением современных технологий программирования и библиотек. Расчёт эволюции динамических переменных выполняется методом Рунге-Кутты~4~порядка. Для оптимизации расчёта электрического поля применено восьмеричное разбиение пространства (частный случай k-дерева для размерности~3) \cite{Bentley1975}.

{\defpositions}
\begin{enumerate}[beginpenalty=10000] % https://tex.stackexchange.com/a/476052/104425
  \item Региональная многопунктовая грозопеленгационная система (ГПС) под названием OpenLDS разработана и введена в эксплуатацию. На протяжении конвективных сезонов 2014--2017\,гг. осуществляется мониторинг молниевой активности на территории Нижегородской и смежных областей. Точность позиционирования молний ГПС составляет около 3\,км.
  
  \item Проведён статистический анализ молниевой активности по данным ГПС на территории Нижегородской области. Количество гроз существенно зависит от типа местности и изменяется более, чем в 2,7 раз на территориях, удалённых на 20\,км, что сравнимо с размером грозового фронта. На основе полученных данных выдвинута гипотеза о влиянии мощных ТЭЦ на снижение количества гроз.
  
  \item Исследованы статистические характеристики повторных компонент обратного удара молнии на основе данных ГПС. Показано, что условная вероятность возникновения каждого последующего повторного обратного удара увеличивается до компоненты~№8, далее остаётся приблизительно неизменной на уровне 0,5, что свидетельствует о том, что средняя многокомпонентная молния не способна нейтрализовать большую часть доступного заряда и что прерывание цепочки повторных разрядов обеспечивается деградацией канала, а не нехваткой заряда в облаке.
  
  \item На основе показаний ГПС определено, что средний интервал между повторными ударами растёт линейно во времени от 76\,мс до 110\,мс, что свидетельствует об усложнении процесса сбора заряда для каждой последующей компоненты.
  
  \item Разработана транспортная самоорганизующаяся численная модель развития молнии на стримерно-лидерной стадии. Модель реалистично описывает процесс инициации разряда от момента возникновения первых децеметровых проводящих областей до устойчивого развития билидера. Модель воспроизводит основные качественные и количественные эффекты, не заложенные явно в её изначальные принципы.
  
  \item На основе моделирования показано, что развитие молниевого разряда на стримерно-лидерной стадии возможно за счёт коллективных эффектов под действием электростатической индукции и не требует привлечения иных механизмов.
\end{enumerate}

{\reliability}
Показания разработанной региональной грозопеленгационной системы OpenLDS верифицированы посредством сравнения с данными допплеровского метеорадиолокатора, а также с показаниями глобальной системы WWLLN. Погрешность позиционирования разряда оценена величиной 3\,км. 

Исследования мультипликативности обратного удара на основе данных ГПС основаны на достаточно мощной выборке, что обеспечивает надёжность статистических характеристик.

Построенная численная модель инициации молнии верифицирована посредством сравнения с результатами натурных наблюдений по данным различных работ. Модель воспроизводит качественные явления, к которым относится реалистичная топология разрядного дерева, асимметрия развития билидера, инверсия заряда на концах лидера вблизи его оси и другие эффекты. Макроскопические параметры модели численно согласуются с известными значениями. Время развития разряда, ток канала лидера, погонный заряд чехла лидера, скорость распространения лидера соответствуют представленным в ряде работ (см. раздел \ref{sec:model-results}). Программный код модели автоматически тестируется на ряде пробных задач при каждом изменении кодовой базы, также применяются регрессионные Unit-тесты.


% {\probation}
% Основные результаты работы докладывались~на:
% перечисление основных конференций, симпозиумов и~т.\:п.

{\contribution}
Автор принимал активное участие в выборе направления исследований по теме диссертации, постановке задач и поиске путей их решения. При выполнении диссертационной работы вклад автор внёс определяющий вклад в построение, настройку и валидацию модели молнии, разработку грозопеленгационной системы и реализацию всего необходимого программного обеспечения для решения поставленных задач. Силами автора также выполнена значительная часть работы по подготовке результатов к публикации.

{\publications} Результаты исследований по теме диссертации опубликованы в 5 статьях в рецензируемых журналах, рекомендованных ВАК для публикации основных материалов \cite{Our2013, MareevWe2016, BulatovMiG, BulatovEnergetik2017, Bulatov2020}. Кроме того, результаты работы докладывались автором и обсуждались на научных семинарах Института прикладной физики РАН, а также на конференциях \cite{rcpl2014,Borok2014,rclp2016, rclp2018,Borok2017,NWP2017} автором лично, и на конференциях \cite{Kuterin2014, Shlyugaev2014, IudinArmenia2017, IudinChina2017} коллегами автора.
