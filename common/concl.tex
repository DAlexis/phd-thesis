%% Согласно ГОСТ Р 7.0.11-2011:
%% 5.3.3 В заключении диссертации излагают итоги выполненного исследования, рекомендации, перспективы дальнейшей разработки темы.
%% 9.2.3 В заключении автореферата диссертации излагают итоги данного исследования, рекомендации и перспективы дальнейшей разработки темы.
\begin{enumerate}
	\item Разработанная и введена в эксплуатацию региональная грозопеленгационная система OpenLDS. Проведено исследование точности системы и оценена погрешность позиционирования молниевых разрядов.
	
	\item На основе данных ГПС показано, что тип местности существенно влияет на грозовую активность в регионе. Выдвинута гипотеза об отрицательном влиянии мощных ТЭЦ на количество гроз. Показано, что максимумы и минимумы грозовой активности могут иметь локальный характер с характерным масштабом в 20\,км.	
	
	\item Проведён статистический анализ молниевой активности по данным ГПС на территории Нижегородской области. Количество гроз существенно зависит от типа местности и изменяется более, чем в 2,7 раз на территориях, удалённых на 20\,км, что сравнимо с размером грозового фронта. На основе полученных данных выдвинута гипотеза о влиянии мощных ТЭЦ на снижение количества гроз.
	
	\item На основе данных ГПС показано, что условная вероятность возникновения каждого последующего повторного обратного удара увеличивается до компоненты~№8, далее остаётся приблизительно неизменной. Данное поведение что свидетельствует о том, что средняя многокомпонентная молния не способна нейтрализовать большую часть доступного заряда и что прерывание цепочки повторных разрядов обеспечивается деградацией канала, а не нехваткой заряда в облаке.
	
	\item На основе данных ГПС показано, что средний интервал между повторными ударами монотонно увеличивается от 76\,мс до 110\,мс, что свидетельствует об усложнении процесса сбора заряда для каждой последующей компоненты.	
	
	\item Создана транспортная самоорганизующаяся модель стримерно-лидерной стадии процесса инициации молнии, обладающая рядом уникальных особенностей по сравнению с уже имеющимися моделями и основанная на оригинальном подходе. Модель корректно описывает качественные и количественные макроскопические процессы, не заложенные явно в её исходные принципы.
	
	\item На основе моделирования показано, что развитие молниевого разряда на стримерно-лидерной стадии возможно за счёт коллективных эффектов под действием электростатической индукции и не требует привлечения иных механизмов.	
\end{enumerate}